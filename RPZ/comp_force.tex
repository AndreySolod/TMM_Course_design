\section{Силовой расчёт механизма}

\subsection{Определение исходных данных, необходимых для силового расчёта механизма}

Исходными данными для силового расчёта механизма являются:

\begin{itemize}
	\item Угловая координата кривошипа $ \varphi_1 = 210^{\circ}$
	\item Угловая скорость $ \omega_1 = 12.8 рад \cdot с^{-1} $
	\item Угловое ускорение $ \varepsilon_1 = -0.495 рад \cdot с^{-2} $
\end{itemize}

\subsection{Нахождение скоростей. План скоростей.}

\begin{equation}
	v_b = \omega_1 \cdot l_1 = 12.8 \cdot 0.082 = 1.052 м/с
\end{equation}

Из плана скоростей:

	$$v_{cb} = 0.912 м/с;$$
	$$v_{c} = 0.453 м/с;$$
	$$v_{s2} = 0.827 м/с;$$
	$$\omega_2 = \dfrac{v_{cb}}{l_bc} = 2.555 рад/c;$$

\subsection{Нахождение ускорений. План ускорений}

Из плана ускорений:

$$a_b = 13.520; $$
$$\varepsilon_2 = 22.004 рад/с^2; $$
$$a_c = 12.618 м/с^2; $$
$$a_{cb} = 6.675 м/с^2; $$
$$a_{s2} = 14.879 м/с^2; $$

\subsection{Определение главных векторов сил инерции и главных моментов сил инерции}

\begin{eqnarray}
	\Phi_{s2} = a_{s2} m_2 = 73.128 H \\
	\Phi_{s3} = a_{s3} m_3 = 174.879 H 
\end{eqnarray}

\begin{eqnarray}
	G_2 = m_2 g = 58.86 H \\
	G_3 = m_3 g = 176.58 H
\end{eqnarray}

\begin{eqnarray}
	M_{\Phi S_1} = J_{1}^{пр} \cdot \varepsilon_1 = 1.16 \cdot 10^4 Н \cdot м\\
	M_{\Phi S_2} = J_{2S} \cdot \varepsilon_2 = 2.617 Н \cdot м
\end{eqnarray}

\subsection{Кинетостатический силовой расчёт механизма}

\subsubsection{Звенья 2--3}

\begin{equation}
	\sum M_c (F_i) = 0
\end{equation}

\begin{equation}
	-M_{\Phi S2} - M_c(G_2) + M(F_{21}^{\tau}) - M(\Phi_{S2}) = 0
\end{equation}

\begin{equation*}
	F_{21}^{\tau} = 65.407 Н
\end{equation*}

\begin{equation}
	\sum F_i = 0
\end{equation}

Из графика:

	$$F_{21}^{n} = 1889668 Н;$$
	$$R_{30} = 217735 Н $$
	$$F_{21} = 1889668.001$$

Тогда:

\begin{equation*}
	F_{23} = 1899233.21 Н
\end{equation*}

\subsubsection{Звено 1.}

\begin{eqnarray}
	\sum M_c (F_i) = 0\\
	\sum F_i = 0
\end{eqnarray}
	$$F_{10} = 1889668 H;$$
	$$M_D^{пр} = 50526.4 H;$$


\subsection{Проверка результатов}

Проверим результаты с использованием программы Jupyter Notebook:

	$$F_{10} = 1899557.660 Н;$$
	$$F_{21} = 1899546.418 Н;$$
	$$F_{32} = 1899454.148 Н;$$
	$$R_{30} = 217759.784 Н;$$
	$$M_D^{пр} = 50156.403 H$$

Погрешность $M_D^{пр}$, определённая в силовом расчёте с использованием программы Jupyter Notebook, по сравнению со средним движущим моментом, найденным в первом листе, можно объяснить пренебрежением приведённого момента $M_{G2}^{пр}$ в первом листе.

\subsection{Погрешность приведённого момента}

Сравнивая приведённый момент, определённый в силовом расчёте, со средним движущим моментом, найденным в первом листе, найдём погрешность:

 \begin{equation}
 	\varepsilon = \dfrac{ \left| M_D^{пр} - M_D^{пр*} \right| }{M_D^{пр}} \cdot 100 \% = \dfrac{\left|50153.4 - 50526.4\right|}{50153.4} \cdot 100 \%  = 0.741 \%
 \end{equation}
 
 \begin{table}
 	\caption{Результаты силового расчёта}
 \begin{tabular}{|c|c|c|c|c|c|}
 	\hline 
 	$F_{10}$, Н & $F_{12}$, Н & $F_{32}$, Н & $R_{30}$, Н &  $M_D^{пр},Н \cdot мм$ & $M_D^{пр*}, H \cdot мм$ \\ 
 	\hline 
 	1889668 & 1889668 & 1899233 & 217735 & 50526.4 & 50153.4 \\ 
 	\hline 
 \end{tabular} 
\end{table}