\section{Проектирование кулачкового механизма}

\subsection{Построение кинематических диаграмм}

По заданному закону движения толкателя определяем $\varphi_п = 188.57^{\circ}$.

В течение полного цикла движения толкатель кулачкорого механизма должен переместиться из начального положения на рпсстояние, соответствующее ходу $h$, а затем возвратиться в исходное положение. Следовательно, для определения $v_1$ и $v_2$, необходимо решить систему:

\begin{equation}
	\left\{\begin{aligned}
	& \int\limits_{\varphi_{нач}}^{\varphi_p} v_{qB} \; d \varphi =  \int\limits_{0}^{\varphi_y} v_{qB} \; d \varphi \\ 
	& S_{\max} = h
	\end{aligned}\right.
\end{equation}

Решая данную систему уравнений, получаем: $$ v_1 = 0.00648 м; \; v_2 = 0.01296 м $$

График ускорения толкателя $a_{qb}$ можно получить, дифференциируя график скорости: $$a_{qb}(\varphi) = \frac{d v_{qb}}{d \varphi}$$

График перемещения толкателя можно получить, интегрируя график скорости: $$ S_{qb} = \int v_{qb} \; d \varphi $$

Графики выполнены в масштабах: $$ \mu_v = 6317 мм/м; \; \mu_a = 1955 мм/м; \; \mu_s = 7644 мм/м$$

\subsection{Определение основных размеров кулачкового механизма}

Основные разеры механизма определяют с помощью фазового портрета, представляющего собой зависимость $S_B(V_{qB})$. Фазовый портрет для механизма с поступательно движущимся толкателем строим методом графического исключения параметра $\varphi_1$ из диаграмм $S_B(\varphi_1), V_{qB}(\varphi_1)$

Ограничивая фазовый портрет лучами, ориентирорванными с учётом $ [ \theta ] $, находим ОДР, внутри которой с учётом левой внеосности назначается положение оси $O_1$ и определяются габаритные размеры кулачка $r_0$.

$$r_0 = 0.0257 м $$

\subsection{Построение профиля кулачка}

Координаты точек профиля кулачка рассчитываются в полярной $rO_1 \psi$ системе координат. Начало координат совпадает с центром вращения кулачка, полярная ось проходит через начальную точку $B_0$ На профиле кулачка.

В полярной системе координат радиус $r_i$ центрового профиля и угло $\psi_i$, определяющий его положение относительно оси, определяют по формулам:

\begin{eqnarray}
	r_i = \sqrt{(S_0 + s_{Bi})^2 + e^2}; \\
	\psi_i = \phi_{1i} - \beta_i; \\
	\beta_i = \arctan{\left( \dfrac{S_0 + S_{Bi}}{e}\right) } - \arctan{\dfrac{S_0}{e}};
\end{eqnarray}

Где $S_0$ --- координата ближней точки толкателя, $S_{Bi}$ --- текущее значение перемещения точки $B$ толкателя, $\phi_{1i}$ --- текущее значение угла поворота кулачка.

Радиус плунжера: $$ r_p = 0.005397 м $$

Построение выполнено в масштабе $\mu_l = 1664 мм/м$

\subsection{Построение графика угла давления}

Построим график зависимости угла давления от положения толкателя. Для этого воспользуемся формулой:

\begin{equation}
	\theta = \arctan{\dfrac{v_{qB}}{S_0 + S_B}}
\end{equation}

График построен в масштабе $3.282 мм/град$