\section{Определение закона движения}

\subsection{Синтез механизма}

По исходным данным в соответствии с условиями работы(2 крайних положения звена 2) определяем длины звеньев механизма:

Длина кривошипа:

\begin{equation}
	l_{ab} = \dfrac{v_{ср}}{4n_1} = \dfrac{0.655}{4 \cdot 2} = 0.082 м
\end{equation}

	Т. к. $\dfrac{l_{bc}}{l_{ab}} = 4.36$, то

\begin{equation}
	l_{bc} = l_{ab} \cdot 4.36 = 0.082 \cdot 4.36 = 0.357 м
\end{equation}

	Т. к. $\dfrac{l_{bs2}}{l_{bc}} = 0.275$, то

\begin{equation}
	l_{bs2} = l_{bc} \cdot 0.275 = 0.357 \cdot 0.275 = 0.098 м.
\end{equation}

\subsection{Построение схемы механизма}

Механизм построен на листе в масштабе $\mu_S = 400 мм/м$. Отрезок $AB = l_{ab} \cdot \mu_S = 32.8 мм$;, отрезок $BC = l_bc \cdot \mu_S = 110 мм$. Угол поворота начального звена разбит на 12 равных интервалов по $30^{\circ}$. 

\subsection{Вычисление передаточных функций}

Средняя угловая скорость звена 1:

\begin{equation}
	\omega_{ср} = 2 \cdot \pi \cdot n_1 = 2 \cdot 3.1415 \cdot 2 = 12.566 рад/с
\end{equation}

Вычисляем передаточные функции с использованием программы Jupyter Notebook.

\begin{equation}
	v_{qc} = \Big | 0.081 \sin{\left (\varphi \right )} + \frac{2.05 \sin{\left (\varphi \right )} \cos{\left (\varphi \right )}}{\sqrt{- 625 \sin^{2}{\left (\varphi \right )} + 11881}} \Big |
\end{equation}

\begin{equation}
	u_{21} = - \frac{0.229 \cos{\left (\varphi \right )}}{\sqrt{- 0.053 \sin^{2}{\left (\varphi \right )} + 1}}
\end{equation}

\begin{table}
\caption{Аналоги скорости точек $S_2; C$; передаточное отношение $u_{21}$}
\begin{tabular}{|c|c|c|c|c|c|c|}
	\hline 
	№& $\varphi$, град & $v_{qs2_x}$, м & $v_{qs2_y}$, м & $v_{qs2}$, м & $v_{qc}$, м & $u_{21}$ \\ 
	\hline 
	0  &    0 &        0 &    0.0594 &  0.0594 &        0 &  -0.229 \\
	\hline 
	1  &   30 &   0.0432 &    0.0514 &  0.0671 &   0.0491 &  -0.200 \\
	\hline 
	2  &   60 &   0.0732 &    0.0297 &  0.0790 &   0.0792 &  -0.117 \\
	\hline 
	3  &   90 &   0.0819 &  6.13e-19 &  0.0819 &   0.0819 &       0 \\
	\hline 
	4  &  120 &   0.0686 &   -0.0297 &  0.0748 &   0.0626 &   0.117 \\
	\hline 
	5  &  150 &   0.0387 &   -0.0514 &  0.0643 &   0.0328 &   0.200 \\
	\hline 
	6  &  180 &        0 &   -0.0594 &  0.0594 &        0 &   0.229 \\
	\hline 
	7  &  210 &  -0.0387 &   -0.0514 &  0.0643 &  -0.0328 &   0.200 \\
	\hline 
	8  &  240 &  -0.0686 &   -0.0297 &  0.0748 &  -0.0626 &   0.117 \\
	\hline 
	9  &  270 &  -0.0819 &  6.13e-19 &  0.0819 &  -0.0819 &       0 \\
	\hline 
	10 &  300 &  -0.0732 &    0.0297 &  0.0790 &  -0.0792 &  -0.117 \\
	\hline 
	11 &  330 &  -0.0432 &    0.0514 &  0.0671 &  -0.0491 &  -0.200 \\
	\hline 
	12 &  360 &        0 &    0.0594 &  0.0594 &        0 &  -0.229 \\
	\hline
\end{tabular} 
\end{table}

\subsection{Построение индикаторных диаграмм $p(S(\varphi))$ и графика сил $F(S(\varphi))$.}

\subsubsection{Построение индикаторной диаграммы}
Индикаторная диаграмма строится по заданной таблице значения давления в цилиндре на поршень. Отрезок хода поршня $h_3 \mu_S$ делим на 12 интервалов. В каждой точке деления строим ординату диаграммы, задавшись максимальной ординатой, равной $81.243 мм$ при $ \dfrac{P}{P_{\max}} = 1 $.

Масштаб индикаторной диаграммы: $ \dfrac{81.243}{2.451} = 33.265 мм/МПа$.

\subsubsection{Построение графика силы $F_c$}

Для определения силы давления $F_c$ на поршень необходимо давление умножить на площадь поршня. Тогда:

Определим площадь поршня:

\begin{equation}
	S_п = \dfrac{\pi d^2}{4} = \dfrac{3.1415 \cdot 0.99^2}{4} = 0.7698 м^2
\end{equation}

График силы $F_c$ строим в масштабе:

\begin{equation}
	\mu_F = \dfrac{\mu_p}{S_п} = \dfrac{33.265}{0.7698} \cdot 10^{-3} = 0.0432 мм/кН 
\end{equation}

\subsection{Построение графиков приведённых моментов движущих сил $ M_д^{пр}(\varphi) $, сил сопротивления $ M_c^{пр}(\varphi) $ и сил тяжести $ M_{G2}^{пр}(\varphi) $}

Для определения закона движения механизма заменяют реальный механизм его одномассовой динамической моделью и находят приложенный к её звену суммарный приведённый момент:

\begin{equation}
	M_{\sum}^{пр} =  M_д^{пр}(\varphi) +  M_c^{пр}(\varphi) 
\end{equation}

\subsubsection{Приведённый момент сопротивления}

К точке $C$ механизма приложена сила сопротивления, равная:

\begin{equation}
	F_c = 
	\left\{\begin{aligned}
	& 37.718 кН, \varphi \in (0 \ldots 180^{\circ})\\ & 1886.703 кН, \varphi \in (180 \ldots 360^{\circ})
	\end{aligned}\right.
\end{equation}

Т. к. работа сил сопротивления за цикл всегда отрицательна, то:

\begin{equation}
	M_{c}^{пр} = F_c \cdot v_{qc} \cdot (-1)
\end{equation}

Приведённым моментом $M_{G2}^{пр}$ сил тяжести $G_2$ пренебрегают, так как он мал по сравнению с моментом $M_c^{пр}$.

График построен в масштабе $$ \mu_{Mпр} = 0.5941 мм/(Н \cdot м) $$

\subsubsection{Приведённый момент движущих сил}

Приведённый момент движущих сил $M_д^{пр}$ определяют из условия, что при установившемся движении $A_д = А_с$ за цикл. Тогда:

\begin{equation}
	A_c = \int_0^{2 \pi}{M_c^{пр}(\varphi) \; d \varphi}
\end{equation}

\begin{equation}
	M_д^{пр}(\varphi) =\text{const} = - \dfrac{A_c}{2 \pi} = 50153.412 Н \cdot м.
\end{equation}

График построен в масштабе $$ \mu_{Mпр} = 0.5941 мм/(Н \cdot м) $$

\subsubsection{Приведённый суммарный момент $M_{\sum}^{пр}$}

\begin{equation}
	 M_{\sum}^{пр} =  M_д^{пр}(\varphi) +  M_c^{пр}(\varphi)
\end{equation}

\begin{table}
\caption{Моменты сопротивления и суммарный приведённый момент}
\begin{tabular}{|c|c|c|c|}
	\hline 
	№& $\varphi$, град & $M_c^{пр}, Н \cdot м$ & $M_{\sum}^{пр}, Н \cdot м$ \\ 
	0  &    0 &           0 &      50153. \\
	\hline 
	1  &   30 &     -92680. &   1.4283e+5 \\
	\hline 
	2  &   60 &  -1.4943e+5 &   1.9959e+5 \\
	\hline 
	3  &   90 &  -1.5447e+5 &   2.0463e+5 \\
	\hline 
	4  &  120 &  -1.1812e+5 &   1.6828e+5 \\
	\hline 
	5  &  150 &     -61794. &   1.1195e+5 \\
	\hline 
	6  &  180 &           0 &      50153. \\
	\hline 
	7  &  210 &      61794. &     -11640. \\
	\hline 
	8  &  240 &   1.1812e+5 &     -67971. \\
	\hline 
	9  &  270 &   1.5447e+5 &  -1.0432e+5 \\
	\hline 
	10 &  300 &   1.4943e+5 &     -99278. \\
	\hline 
	11 &  330 &      92680. &     -42527. \\
	\hline 
	12 &  360 &           0 &      50153. \\
	\hline 
\end{tabular} 

\end{table}

График построен в масштабе $$ \mu_{Mпр} = 0.000346 мм/(Н \cdot м) $$

\subsection{Приведённые моменты инерции звеньев \RNumb{2} группы}

Приведённые моменты инерции определяют по формуле:

\begin{equation}
	J_{i}^{пр} = J_{is} \cdot \omega_{qi}^2 + m_i \cdot v_{qi}^2.
\end{equation}

Где:

$ J_{i}^{пр} $ --- приведённый момент инерции $i$--ого звена

$ J_{is} $ --- Момент инерции $ i $--го звена относительно центра масс

$ \omega_{qi} $ --- аналог угловой скорости $ i $--го звена

$ m_i $ --- масса $ i $--го звена.

$  v_{qi} $ --- аналог линейной скорости центра масс $ i $--го звена.

По данной формуле расчитываем $J_2^{пр}; \; J_3^{пр}; \; J_{\sum}^{пр}$.

\begin{equation}
	J_2^{пр} = J_{2s} \cdot \omega_{q2}^2 + m_2 \cdot v_{q2}^2
\end{equation}

\begin{equation}
J_3^{пр} = m_2 \cdot v_{q2}^2
\end{equation}

\begin{equation}
J_{\sum}^{пр} = J_2^{пр} + J_3^{пр}
\end{equation}

\begin{table}
\caption{Приведённые моменты 2 группы звеньев}
\begin{tabular}{|c|c|c|c|c|}
	\hline 
	№ & $\varphi$, град & $J_2^{пр}, кг \cdot м^2$ &  $J_3^{пр}, кг \cdot м^2$ &  $J_{\sum}^{пр}, кг \cdot м^2$ \\ 
	\hline
	0  &    0 &  0.021878 &          0 &  0.021878 \\
	\hline
	1  &   30 &  0.040574 &    0.12650 &   0.16708 \\
	\hline
	2  &   60 &  0.023149 &  0.0068385 &  0.029987 \\
	\hline
	3  &   90 &  0.034715 &   0.077249 &   0.11196 \\
	\hline
	4  &  120 &  0.029512 &   0.057449 &  0.086961 \\
	\hline
	5  &  150 &  0.033141 &   0.083345 &   0.11649 \\
	\hline
	6  &  180 &  0.031716 &   0.057328 &  0.089044 \\
	\hline
	7  &  210 &  0.024938 &   0.016729 &  0.041667 \\
	\hline
	8  &  240 &  0.039808 &    0.12500 &   0.16481 \\
	\hline
	9  &  270 &  0.022607 &  0.0056206 &  0.028228 \\
	\hline
	10 &  300 &  0.040097 &    0.11935 &   0.15945 \\
	\hline
	11 &  330 &  0.022114 &  0.0012621 &  0.023376 \\
	\hline
	12 &  360 &  0.037404 &   0.096645 &   0.13405 \\
	\hline
	
\end{tabular} 
\end{table}

График построен в масштабе $\mu_j = 424 мм/кг \cdot м^2$

\subsection{Суммарная работа}

Суммарная работа всех сил равна работе $M_{\sum}^{пр}$:

\begin{equation}
	A_{\sum} = \int{M_{\sum}^{пр} \; d \varphi}
\end{equation}

График построен в масштабе: $$\mu_A = 0.219 мм/кДж$$

\subsection{Кинетическая энергия \RNumb{2} группы звеньев}

\begin{equation}
	T_{2}^{пр} = J_{\sum}^{пр} \cdot \dfrac{\omega_{ср}^2}{2}
\end{equation}

\subsection{Изменение кинетической энергии \RNumb{1} группы звеньев}

\begin{equation}
	\Delta T_1(\varphi) = A_{\sum}(\varphi) - T_2^{пр}(\varphi)
\end{equation}

\begin{equation*}
	\Delta T_{1}^{нб} = 169 кДж 
\end{equation*}

\subsection{Кинетическая энергия \RNumb{1} группы звеньев}

Кинетическая энергия \RNumb{1} группы звеньев находится из зависимости 

\begin{equation}
	T_{1} = T - T_2
\end{equation}

Графи построен в масштабе $$\mu_T = 0.524 мм/кДж$$

\subsection{Необходимый момент инерции маховых масс $J_1^{пр}$}

\begin{equation}
	J_1^{пр} = \dfrac{\Delta T_1^{нб} - \delta(T_{2Q} - T_{2N})}{\omega_{ср}^2 \cdot \delta}
\end{equation}

\begin{equation*}
	J_1^{пр} = 2.349 \cdot 10^4 кг \cdot м^2
\end{equation*}

\subsection{Угловая скорость звена приведения}

\begin{equation}
	\Delta \omega(\varphi) = \dfrac{\Delta T_1(\varphi) - \dfrac{T_{\max} + T_{\min}}{2}}{\omega_{ср} \cdot J_1^{пр}}
\end{equation}

График построен в масштабе $$ \mu_{\omega} = 151.926 мм/(рад \cdot с^{-1}) $$

\subsection{Расчёт габаритных размеров и массы маховика}

Маховик может быть выполнен:

\begin{itemize}
	\item в форме сплошного диска
	\item в форме обода со шлицами и ступицей
\end{itemize}

В осевом сечении обод маховика имеет форму прямоугольника, стороны которого ограничиваются наружным $D_2$, внутренним $ D_1 $ диаметрами и толщиной $b$. Соотношения между размерами: $$ \psi_b = \dfrac{b}{D_2}; \psi_h = \dfrac{D_1}{D_2}$$

При $ \psi_b = 0.2 $ и $ \psi_h = 0.8$:

\subsubsection{В форме сплошного диска}

\begin{equation}
	D = 0.366 \cdot \sqrt[5]{J_{доп}} = 2.74 м
\end{equation}

\begin{equation}
	b = 0.2D = 0.548 м
\end{equation}

\begin{equation}
	m = 1230 D^3 = 2.529 \cdot 10^4 кг
\end{equation}

\subsubsection{В форме обода}

\begin{equation}
	D_2 = 0.437 \sqrt[5]{J_{доп}} = 3.271 м
\end{equation}

\begin{equation}
D_1 = 0.8  D_2 = 2.617 м
\end{equation}

\begin{equation}
b = 0.2  D_2 = 0.654 м
\end{equation}

\begin{equation}
m = 6123(D_2^2 - D_1^2)b = 15428.546 кг
\end{equation}