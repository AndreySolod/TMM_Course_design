\section{Проектирование зубчатой передачи и планетарного редуктора}

\subsection{Выбор коэффициента смещения реечного инструмента}

Производим вычисление эвольвентной зубчатой передачи $z_1, z_2$ на программе (3list.xls). Результаты работы программы представлены в табл. \ref{table:3list} 

\begin{table}[h]
	
\caption{Результаты работы программы 3list.xls}

\begin{tabular}{|c|c|c|c|}
	\hline 
	$z_1 = 10$ & $z_2 = 19$ & $m = 4.0$ & $\beta = 25^{\circ}$ \\ 
	\hline 
	$\alpha = 20$ & $h_{a} = 1$ & $c = 0.25$ & $a_{w0} = 0$ \\ 
	\hline 
	\multicolumn{4}{|c|}{Результаты расчёта} \\ 
	\hline 
	$x_2 = 0.5$ & $r_{1} = 22.065$ & $r_{2} = 41.924$ & $r_{b1} = 20.478$ \\ 
	\hline 
	$r_{b2} = 38.905$ & $p_t = 13.858$ & $m_t = 4.4135$ & $h_{at} = 0.9063$ \\ 
	\hline 
	$c_t = 0.2266$ & $\alpha_t = 21.88$ & $\rho = 1.52$ & $p_{1x} = 13.693$ \\ 
	\hline 
	$p_{2x} = 13.802$ & $z_{\min t} = 13.052$ & $x_{\min 1t} = 0.2119$ & $x_{\min 2t} = -0.413$ \\ 
	\hline 
	$S_o = 6.9292$ &  &  &  \\ 
	\hline 
\end{tabular}
\label{table:3list}

\end{table}

Выбираем коэффициент смещения из общих требований:

\begin{enumerate}
	\item Коэффициент перекрытия проектируемой передачи должен быть больше допустимого ($ \varepsilon_{\alpha} > [\varepsilon_{\alpha}] $);
	\item Зубья у проектируемой передачи не должны быть подрезаны, и толщина их по окружности вершин должна быть больше допустимой($ S_a > [S_a] $)
\end{enumerate}

Принимаем коэффициент смещения $x_1 = 0.4$.

\begin{table}[h]
	\caption{Значения параметров зубчатого колеса при $x_1 = 0.5$}
	\begin{tabular}{|c|c|c|c|}
		\hline 
		$y_1 = 0.791$ & $dy = 0.109$ &  $r_{w1} = 23.271$ & $r_{w2} = 44.231$ \\ 
		\hline 
		$a_w = 67.503$ & $r_{a1} = 27.351$ & $r_{a2} = 47.653$ & $r_{f1} = 18.833$ \\ 
		\hline 
		$r_{f2} = 39.135$ & $h = 8.518$ & $s_1 = 8.351$ & $s_2 = 8.705$ \\ 
		\hline 
		$\alpha_{wt} = 28.375$ & $s_{a1} = 2.638$ & $s_{a2} = 3.034$ & $\varepsilon_{\alpha} = 1.12$ \\ 
		\hline 
		$\varepsilon_{\gamma} = 1.381$ & $\lambda_1 = 2.179$ & $\lambda_2 = 0.776$ & $\theta = 0.0.609$ \\ 
		\hline 
	\end{tabular} 
	\label{final_table}
\end{table}

\subsection{Построение профиля колеса, изготовляемого реечным инструментом}

Выбираем масштаб построения $\mu_l = 5 мм/мм$

Шаг зубьев по делительной прямой ИПК: $p = \pi m = 12.566мм$

%Радиус кривизны переходной кривой зуба $\rho_f = \dfrac{c* m}{1 - \sin{\alpha}} = 

Были проведены окружности:

\begin{enumerate}
	\item Делительная: $l_{r1} = r_1 \cdot \mu_l = 110.325 мм$
	\item Основная: $l_{rb1} = r_{b1} \cdot \mu_l = 102.39 мм$
	\item Вершин: $l_{ra1} = r_{a1} \cdot \mu_l = 138.53 мм$
	\item Впадин: $l_{rf1} = r_{f1} \cdot \mu_l = 96.37 мм$
\end{enumerate}

Отложено от делительной окружности выбранное смещение $x_1 \cdot m_t \cdot \mu_l = 121.370 мм$, и проведена делительная прямая исходного производящего контура реечного инструмента. На расстоянии $h_a \cdot m \cdot \mu_l = 20 мм$ вверх и вниз от делительной прямой проведены прямые граничных точек, а на расстоянии $h_a \cdot m \cdot \mu_l + c* \cdot m \cdot \mu_l = 25мм$ --- прямые вершин и впадин. Станочно--начальную прямую проводим касательно к делительной окружности в точке $P_0$(полюс станочного зацепления). Была проведена линия станочного зацепления $N_1 P_0$ через полюс станочного зацепления $P_0$ касательно к основной окружности в точке $N_1$. Построен исходный производящий контур реечного инструмента так, чтобы ось симметрии впадины совпадала с вертикалью. Для этого от точки пересечения вертикали с делительной прямой откладывается влево по горизонтали отрезок в $0.25$ шага, равный $17.137$мм, и через конец его перпендикулярно линии зацепления $N_1 P_0$ проводится наклонная прямая, которая образует угол с вертикалью. Эта прямая является прямолинейной частью профиля зуба исходного производящего контура инструмента. Закруглённый участок профиля построен как сопряжение прямолинейной части контура с прямой вершин или с прямой впадин окружностью радиусом $l_{\rho f} = \rho_f \cdot \mu_l = 7.6 мм$. Симметрично относительно вертикали(линия симметрии впадин) построен профиль второго зуба исходного производящего контура, прямолинейный участок которого перпендикулярен к другой возможной линии зацепления. Расстояние между одноименными профилями зубьев исходного контура равно шагу $p = \pi \cdot m \cdot \mu_l = 68.551 мм$. 

Построен профиль зуба прокетируемого колеса, касающийся профиля исходного производящего контура в точке $K$. Для построения ряда последовательных положений профиля зуба исходного производящего контура проведена вспомогательная прямая касательно к окружности вершин. Откладываем на начальной прямой отрезки, равные 20 мм. Такие же отрезки отложены на станочно--начальной прямой и на дуге делительной окружности. Из центра колеса через точки дуги делительной окружности проведены лучи до пересечения с окружностью вершин. При перекатывания без скольжения станочно--начальной прямой по делительной окружности точки на станочно--начальной прямой и точки  и точки на дуге делительной окружности последовательно совпадают; то же для точек на начальной прямой и точек на окружности вершин. При этом точка $W$ описывает укороченную эвольвенту, а точка $L$ --- удлинённую.

\subsection{Построение проектируемой зубчатой предачи}

Откладываем межосевое расстояние $l_{aw} = a_w \cdot \mu_l = 472.395 мм$. 

Проведены окружности:

\begin{enumerate}
	\item Начальная $l_{rw2} = r_{w2} \cdot \mu_l = 217.869 мм$
	\item Делительная: $l_{r21} = r_2 \cdot \mu_l = 205.427 мм$
	\item Основная: $l_{rb2} = r_{b2} \cdot \mu_l = 197.64 мм$
	\item Вершин: $l_{ra2} = r_{a2} \cdot \mu_l = 230.616 мм$
	\item Впадин: $l_{rf2} = r_{f2} \cdot \mu_l = 191.147 мм$
\end{enumerate}

Через полюс зацепления касательно к основным окружностям колёс проведены линии зацепления.

На каждом колесе построены профили зубьев. Профили зубьев шестерни строим по ранее сделанному шаблону для станочного звацепления, эволвентные профили зубьев колеса строим ка траекторию точки примой при перекатывании ее по основной окружности колеса без скольжения. Для этого, через равные углы проведены лучи из центра $O_2$ до пересечения с основной окружностью. Были проведены касательные к основной окружности в точках пересечения. Из точек пересечения радиусом равным расстоянию до первой точки пересечения сделаны засечки на касательных. Последовательно соединяя полученнче засечки, получаем левую половину эвольвентного профиля зуба. Переносим полученную эвольвенту в точку контакта зубьев $K$ на линию зацепления. Т. к. $r_{f2} < r_{b2}$, но $ (r_{b2} - r_{f2}) \leqslant 0.4 \cdot m $, то сопрягаем эвольвентную часть профиля с окружностью впадин. От построенного профиля зуба откладываем толщину зуба по делительной окружности, по окружности вершин, и проводим аналогичный профиль другой стороны.

\subsection{Проектирование планетарного зубчатого механизма с цилиндрическими колёсами}

Передаточное отношение для планетарного редуктора $U_{1H} = 9.4$, число сателлитов в планетарном редукторе $k = 3$.

%Исходя из этих данных выбираем двухрядный планетарный механизм со смешанным зацеплением.

\subsection{Кинематический синтез двухрядного планетарного механизма}

Для двухрядного планетарного механизма должны выполняться следующие условия:

\begin{enumerate}
	\item $ U_{1H} = 1 + \dfrac{z_2 z_4}{z_1 z_3}$
	\item $z_1 \geqslant 17; z_2 \geqslant 17; z_3 \geqslant 20; z_4 \geqslant 85$
	\item $z_1 + z_2 = z_4 - z_3$
	\item $\sin{\frac{\pi}{k}} > \frac{\max(z_2, z_3) + 2 h_a*}{z_1 + z_2}$
	\item $ \frac{z_1 U_{1H}}{k}(1 + kП) = Ц$
\end{enumerate}

Для определения $z_1, z_2, z_3, z_4$ воспользуемся программой planet.py(Приложение А).

Результаты работы программы planet.py:

$z_1 = 18; z_2 = 54; z_3 = 40; z_4 = 112$

Тогда диаметры колёс:

\begin{enumerate}
	\item $d_1 = mz_1 = 18 мм$
	\item $d_2 = mz_2 = 54 мм$
	\item $ d_3 = mz_3 = 40 мм $
	\item $ d_4 = mz_4 = 112 мм $
\end{enumerate}

Масштаб графика $\mu_l = 1 мм/мм$