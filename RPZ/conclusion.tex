\section*{Заключение}
\addcontentsline{toc}{section}{Заключение}

В ходе выполнения курсовой работы были получены следующие результаты:

\begin{enumerate}
	\item В процессе курсового проетирования был установлен закон движения основного механизма плунжерного насоса простого действия. Были установлены зависимость $\omega(t), \varepsilon(t)$, рассчитан необходимый момент инерции маховых масс, обеспечивающий заданный коэффициент неравномерности $\delta = \frac{1}{22}$
	
	\item Для заданного положения механизма $\varphi_1 = 210^{\circ}$ проведён силовой расчёт, определены реакции в кинематических парах механизма и движущий момент. Величина этого момента отличается от движущего момента, полученного на первом листе на 0.741 \%
	
	\item Спроектирована прямозубая цилиндрическая эвольвентная зубчатая передача с модулем $m = 4.0$, с числом зубьев колёс $z_1 = 10, z_2 = 19$, коэффициентами смещения $x_1 = 0.5, x_2 = 0.5$ и коэффициентом перекрытия $\varepsilon_{\gamma} = 1.296$
	
	\item Спроектирован планетарный редуктор с передаточым отношением $U_{1H} = 9.4$ с $z_1 = 18; z_2 = 54; z_3 = 40; z_4 = 112$.
	
	\item Спроектирован кулачковый механизм с поступательно движущимся толкателем при заданном законе движения толкателя. Минимальный теоретический радиус кулачка $r_0 = 0.0257 м$, радиус плунжера $r_{p} = 0.005397 м$, при допустимом угле давления $\theta = 24^{\circ}$
\end{enumerate}